\documentclass[a4paper]{article}
\usepackage{ctex}
\ctexset{
    proofname = \heiti{证明}
}
\usepackage{amsmath, amssymb, amsthm}
\usepackage{float} % 提供[H]固定位置选项
\usepackage{moreenum}
\usepackage{mathtools}
\usepackage{url}
\usepackage{bm}
\usepackage{enumitem}
\usepackage{graphicx} % 只保留一次加载
\usepackage{subcaption}
\usepackage{booktabs}
\usepackage{listings}
\usepackage{xcolor}
\usepackage{setspace}
\usepackage{geometry}
\usepackage[mathcal]{eucal}
\usepackage[thehwcnt = 1]{iidef}

% 设置图片文件夹路径(建议使用相对路径,避免绝对路径移植问题)
\graphicspath{{c:/Users/吴虹霖/Desktop/2025小学期/系统开发工具基础/一些图片/}}

% 页面边距配置
\geometry{a4paper, top=2.5cm, bottom=2.5cm, left=3cm, right=2.5cm}
% 全局行间距1.2倍
\linespread{1.2}


\thecourseinstitute{中国海洋大学信息科学与工程学部}
\thecoursename{系统开发工具基础}
\theterm{2025年夏季学期}
\hwname{第一周实验报告}
\slname{\heiti{解}}

\begin{document}
\courseheader
\name{24计算机科学与技术 吴虹霖 \quad 学号: 24020007135}

% ------------------- 第一大板块:Git 练习实例 -------------------
\section{一、Git 练习实例}
\begin{enumerate}[itemsep=2\parskip, label=实例1.\arabic*]

    % 实例1.1:Git命令和数据模型相关内容
\item \textbf {Git 命令和数据模型相关内容}
\begin{description}[
leftmargin=7em,
labelwidth=5em,
labelsep=1em,
itemsep=1\parskip,
align=right
]
\item [\textbf {Q}:] 如果您之前从来没有用过 Git,推荐您阅读 Pro Git 的前几章,或者完成像 Learn Git Branching 这样的教程。重点关注 Git 命令和数据模型相关内容;
\item [\textbf {A}:] 1. 数据模型:Git 通过 “工作区→暂存区→本地版本库” 三层结构管理代码,分支本质是指向特定提交记录的引用,所有历史版本通过提交对象链串联,形成可追溯的变更记录。\\
2. 核心命令:围绕数据流转和分支操作,\texttt {git add} 将工作区文件移至暂存区,\texttt {git commit} 将暂存区内容提交到版本库形成历史记录;\texttt {git branch} 创建分支,\texttt {git checkout} 切换分支,\texttt {git merge} 合并分支,支持多线开发与历史整合。
\end{description}

% 实例1.2:历史版本可视化
\item \textbf {历史版本可视化} % 修正了标题重复问题
\begin{description}[
leftmargin=7em,
labelwidth=5em,
labelsep=1em,
itemsep=1\parskip,
align=right
]
\item [\textbf {Q}:] 克隆本课程网站的仓库,将版本历史可视化并进行探索
\item [\textbf {A}:] 可视化版本历史
\begin{figure}[H]
  \centering
  \includegraphics[width=0.8\textwidth]{查看历史版本.png}
  \caption{查看历史版本截图}
  \label{fig:view-history} % 唯一标签
\end{figure} 
\end{description}

    % 实例1.3:查看修改
\item \textbf {查看修改}
\begin{description}[
leftmargin=7em,
labelwidth=5em,
labelsep=1em,
itemsep=1\parskip,
align=right
]
\item [\textbf {Q}:] 是谁最后修改了 README.md 文件?(提示:使用 git log 命令并添加合适的参数)
\item [\textbf {A}:] 如图所示。
\begin{figure}[H]
  \centering
  \includegraphics[width=0.8\textwidth]{查看修改人.png}
  \caption{查看修改人截图}
  \label{fig:view-modifier} % 唯一标签
\end{figure}
\end{description}

    % 实例1.4:查看文件修改与详细信息
\item \textbf {查看文件修改与详细信息}
\begin{description}[
leftmargin=7em,
labelwidth=5em,
labelsep=1em,
itemsep=1\parskip,
align=right
]
\item [\textbf {Q}:] 最后一次修改\_config.yml 文件中 collections: 行时的提交信息是什么?(提示:使用 git blame 和 git show)
\item [\textbf {A}:] 如图所示。
\begin{figure}[H] % 修正为figure环境(原itemize错误)
  \centering
  \includegraphics[width=0.8\textwidth]{查看最后一次修改记录.png}
  \caption{查看最后一次修改记录截图}
  \label{fig:last-modify-record} % 唯一标签
\end{figure}

\begin{figure}[H] % 修正为figure环境(原itemize错误)
  \centering
  \includegraphics[width=0.8\textwidth]{查看提交详细信息.png}
  \caption{查看提交详细信息截图}
  \label{fig:commit-details} % 唯一标签
\end{figure}
\end{description}

    % 实例1.5:提交添加信息并删除
\item \textbf {提交添加信息并删除}
\begin{description}[
leftmargin=7em,
labelwidth=5em,
labelsep=1em,
itemsep=1\parskip,
align=right
]
\item [\textbf {Q}:] 使用 Git 时的一个常见错误是提交本不应该由 Git 管理的大文件,或是将含有敏感信息的文件提交给 Git 。尝试向仓库中添加一个文件并添加提交信息,然后将其从历史中删除
\item [\textbf {A}:] 如图所示。
\begin{figure}[H]
  \centering
  \includegraphics[width=0.8\textwidth]{添加敏感信息.png}
  \caption{添加敏感信息}
  \label{fig:view-modifier} % 唯一标签
\end{figure}

\begin{figure}[H]
  \centering
  \includegraphics[width=0.8\textwidth]{清理提交信息.png}
  \caption{清理提交信息.png}
  \label{fig:view-modifier} % 唯一标签
\end{figure}
\end{description}

    % 实例1.6:git stash
\item \textbf {git stash}
\begin{description}[
leftmargin=7em,
labelwidth=5em,
labelsep=1em,
itemsep=1\parskip,
align=right
]
\item [\textbf {Q}:] 从 GitHub 上克隆某个仓库,修改一些文件。当您使用 git stash 会发生什么?
\item [\textbf {A}:] 执行 git stash 后,添加到暂存区的内容不会再提示需要提交(Changes to be committed)。且HEAD引用不会变动。
\end{description}

% 实例1.7:别名
\item \textbf {别名}
\begin{description}[
leftmargin=7em,
labelwidth=5em,
labelsep=1em,
itemsep=1\parskip,
align=right
]
\item [\textbf {Q}:] 请在 ~/.gitconfig 中创建一个别名,使您在运行 git graph 时,您可以得到 git log --all --graph --decorate --oneline 的输出结果
\item [\textbf {A}:] 如图所示。
\begin{figure}[H]
  \centering
  \includegraphics[width=0.8\textwidth]{添加别名.png}
  \caption{添加别名截图}
  \label{fig:view-modifier} % 唯一标签
\end{figure}

\begin{figure}[H]
  \centering
  \includegraphics[width=0.8\textwidth]{输入gitgraph.png}
  \caption{输入git graph截图}
  \label{fig:view-modifier} % 唯一标签
\end{figure}
\end{description}

% 实例1.8:git init
\item \textbf {git init}
\begin{description}[
leftmargin=7em,
labelwidth=5em,
labelsep=1em,
itemsep=1\parskip,
align=right
]
\item [\textbf {Q}:] 使用git init初始化仓库
\item [\textbf {A}:] 如图所示。
\begin{figure}[H]
  \centering
  \includegraphics[width=0.8\textwidth]{初始化仓库.png}
  \caption{初始化仓库截图}
  \label{fig:view-modifier} % 唯一标签
\end{figure}
\end{description}

% 实例1.9:克隆仓库
\item \textbf {克隆仓库}
\begin{description}[
leftmargin=7em,
labelwidth=5em,
labelsep=1em,
itemsep=1\parskip,
align=right
]
\item [\textbf {Q}:] 使用git clone克隆一个仓库
\item [\textbf {A}:] 如图所示。
\begin{figure}[H] % 使用[H]固定位置,避免浮动
  \centering
  \includegraphics[width=0.8\textwidth]{克隆仓库.png}
  \caption{克隆仓库截图}
  \label{fig:clone-repo} % 唯一标签
\end{figure}  
\end{description}

% 实例1.10:git status
\item \textbf {git status}
\begin{description}[
leftmargin=7em,
labelwidth=5em,
labelsep=1em,
itemsep=1\parskip,
align=right
]
\item [\textbf {Q}:] 使用git status查看文件状态
\item [\textbf {A}:] 如图所示。
\begin{figure}[H] % 使用[H]固定位置,避免浮动
  \centering
  \includegraphics[width=0.8\textwidth]{git status.png}
  \caption{git status.png}
  \label{fig:clone-repo} % 唯一标签
\end{figure}  
\end{description}
\end{enumerate}


% ------------------- 第二大板块:LaTeX 使用 -------------------
\section{二、LaTeX 使用}
\begin{enumerate}[itemsep=2\parskip, label=实例2.\arabic*]

% 实例2.1:基本文档结构
\item \textbf{基本文档结构}
\begin{description}[
leftmargin=7em,
labelwidth=5em,
labelsep=1em,
itemsep=1\parskip,
align=right
]
\item [\textbf{Q}:] 创建一个简单的LaTeX文档,包含标题、作者、日期和章节结构
\item [\textbf{A}:] \begin{lstlisting}[
    language={[LaTeX]TeX},
    caption={一个简单的LaTeX文档示例},
    label=code:latex-example,
    basicstyle=\ttfamily\small,
    breaklines=true,
    frame=single,
    numbers=left,
    numberstyle=\tiny\color{gray},
    keywordstyle=\color{blue},
    commentstyle=\color{green!50!black},
    stringstyle=\color{red},
    showstringspaces=false,
    escapeinside={(*}{*)}
]
\documentclass{article}

\title{一个简单的LaTeX文档}
\author{姓名}
\date{\today}

\begin{document}

\maketitle

\section{引言}
这是引言部分的内容。

\section{方法}
这里描述所用的方法。

\section{结果}
这里展示实验结果。

\section{结论}
这里给出结论和未来展望。

\end{document}
\end{lstlisting}
\end{description}

% 实例2.2:数学公式排版
\item \textbf{数学公式排版}
\begin{description}[
leftmargin=7em,
labelwidth=5em,
labelsep=1em,
itemsep=1\parskip,
align=right
]
\item [\textbf{Q}:] 使用LaTeX排版包含积分、求和和矩阵的复杂数学公式
\item [\textbf{A}:] 
\begin{lstlisting}[
    language={[LaTeX]TeX},
    caption={数学公式排版},
    label=code:latex-example,
    basicstyle=\ttfamily\small,
    breaklines=true,
    frame=single,
    numbers=left,
    numberstyle=\tiny\color{gray},
    keywordstyle=\color{blue},
    commentstyle=\color{green!50!black},
    stringstyle=\color{red},
    showstringspaces=false,
    escapeinside={(*}{*)}
]
\documentclass{article}

\[
I = \int_{0}^{1} \left( \sum_{n=1}^{\infty} \frac{1}{n^2} \right) \cdot
\begin{pmatrix}
a & b \\
c & d
\end{pmatrix}
\, dx
\]
\end{lstlisting}
\end{description}

% 实例2.3:表格制作
\item \textbf{表格制作}
\begin{description}[
leftmargin=7em,
labelwidth=5em,
labelsep=1em,
itemsep=1\parskip,
align=right
]
\item [\textbf{Q}:] 创建一个三线表并添加表格标题和标签
\item [\textbf{A}:]
\begin{lstlisting}[
    language={[LaTeX]TeX},
    caption={表格制作},
    label=code:latex-example,
    basicstyle=\ttfamily\small,
    breaklines=true,
    frame=single,
    numbers=left,
    numberstyle=\tiny\color{gray},
    keywordstyle=\color{blue},
    commentstyle=\color{green!50!black},
    stringstyle=\color{red},
    showstringspaces=false,
    escapeinside={(*}{*)}
]
\documentclass{article}

\[\begin{table}[H]
  \centering
  \caption{学生成绩统计表}
  \label{tab:score-table}
  \begin{tabular}{lccc}
    \toprule
    姓名 & 语文 & 数学 & 英语 \\
    \midrule
    张三 & 90 & 95 & 88 \\
    李四 & 85 & 80 & 92 \\
    王五 & 78 & 89 & 84 \\
    \bottomrule
  \end{tabular}
\end{table}
]
\end{lstlisting}

\end{description}

% 实例2.4:图片插入与排版
\item \textbf{图片插入与排版}
\begin{description}[
leftmargin=7em,
labelwidth=5em,
labelsep=1em,
itemsep=1\parskip,
align=right
]
\item [\textbf{Q}:] 插入图片并实现图文混排效果
\item [\textbf{A}:] 
\begin{lstlisting}[
    language={[LaTeX]TeX},
    caption={图片插入与排版},
    label=code:latex-example,
    basicstyle=\ttfamily\small,
    breaklines=true,
    frame=single,
    numbers=left,
    numberstyle=\tiny\color{gray},
    keywordstyle=\color{blue},
    commentstyle=\color{green!50!black},
    stringstyle=\color{red},
    showstringspaces=false,
    escapeinside={(*}{*)}
]

\documentclass{article}

\[\begin{figure}[H]
  \centering
  \includegraphics[width=0.5\textwidth]{example-image} % 替换为实际图片文件名
  \caption{示例图片}
  \label{fig:example-image}
]
\end{lstlisting}

\end{description}

% 实例2.5:参考文献管理
\item \textbf{参考文献管理}
\begin{description}[
leftmargin=7em,
labelwidth=5em,
labelsep=1em,
itemsep=1\parskip,
align=right
]
\item [\textbf{Q}:] 使用BibTeX管理并引用参考文献
\item [\textbf{A}:] 
\begin{lstlisting}[
    language={[LaTeX]TeX},
    caption={参考文献管理},
    label=code:latex-example,
    basicstyle=\ttfamily\small,
    breaklines=true,
    frame=single,
    numbers=left,
    numberstyle=\tiny\color{gray},
    keywordstyle=\color{blue},
    commentstyle=\color{green!50!black},
    stringstyle=\color{red},
    showstringspaces=false,
    escapeinside={(*}{*)}
]
\documentclass{article}

\[\\documentclass{beamer}

% 设置参考文献
\usepackage{natbib}
\bibliographystyle{plain}

\title{简单BibTeX示例}
\begin{document}

\begin{frame}{参考文献引用}
    这是一个引用示例\cite{book1}。
    
    这是另一个引用示例\cite{article1}。
\end{frame}

\begin{frame}{参考文献列表}
    \bibliography{refs} % 引用BibTeX文件
\end{frame}

\end{document}

@book{book1,
  title={简单书籍},
  author={作者一},
  year={2020},
  publisher={出版社}
}

@article{article1,
  title={简单文章},
  author={作者二},
  journal={期刊名},
  year={2021}
}
]
\end{lstlisting}

\end{description}

% 实例2.6:自定义命令与环境
\item \textbf{自定义命令与环境}
\begin{description}[
leftmargin=7em,
labelwidth=5em,
labelsep=1em,
itemsep=1\parskip,
align=right
]
\item [\textbf{Q}:] 定义新的命令和环境以提高文档编写效率
\item [\textbf{A}:] 
\begin{lstlisting}[
    language={[LaTeX]TeX},
    caption={自定义命令与环境},
    label=code:latex-example,
    basicstyle=\ttfamily\small,
    breaklines=true,
    frame=single,
    numbers=left,
    numberstyle=\tiny\color{gray},
    keywordstyle=\color{blue},
    commentstyle=\color{green!50!black},
    stringstyle=\color{red},
    showstringspaces=false,
    escapeinside={(*}{*)}
]
\documentclass{article}
\[\documentclass{beamer}

% 定义一个简单的自定义命令
\newcommand{\hi}{\textcolor{red}{重要!}}

% 定义一个简单的自定义环境
\newenvironment{mybox}
  {\begin{block}{注意}}
  {\end{block}}

\begin{document}

\begin{frame}{简单自定义示例}
    这是一个\hi 自定义命令的例子。
    
    \begin{mybox}
        这是一个自定义环境的例子。
    \end{mybox}
\end{frame}

\end{document}]
]
\end{lstlisting}
\end{description}

% 实例2.7:列表与枚举
\item \textbf{列表与枚举}
\begin{description}[
leftmargin=7em,
labelwidth=5em,
labelsep=1em,
itemsep=1\parskip,
align=right
]
\item [\textbf{Q}:] 创建多级列表并自定义列表样式
\item [\textbf{A}:] 
\begin{lstlisting}[
    language={[LaTeX]TeX},
    caption={列表与枚举},
    label=code:latex-example,
    basicstyle=\ttfamily\small,
    breaklines=true,
    frame=single,
    numbers=left,
    numberstyle=\tiny\color{gray},
    keywordstyle=\color{blue},
    commentstyle=\color{green!50!black},
    stringstyle=\color{red},
    showstringspaces=false,
    escapeinside={(*}{*)}
]
\documentclass{article}
\[\documentclass{beamer}
\begin{document}

\begin{frame}{简单列表}
    \begin{itemize}
        \item 主要项目
        \begin{itemize}
            \item 子项目
            \item 子项目
        \end{itemize}
        \item 主要项目
    \end{itemize}
\end{frame}

\end{document}
]
\end{lstlisting}

\end{description}

% 实例2.8:字体与颜色设置
\item \textbf{字体与颜色设置}
\begin{description}[
leftmargin=7em,
labelwidth=5em,
labelsep=1em,
itemsep=1\parskip,
align=right
]
\item [\textbf{Q}:] 设置文档中不同部分的字体样式和颜色
\item [\textbf{A}:] 
\begin{lstlisting}[
    language={[LaTeX]TeX},
    caption={字体与颜色设置},
    label=code:latex-example,
    basicstyle=\ttfamily\small,
    breaklines=true,
    frame=single,
    numbers=left,
    numberstyle=\tiny\color{gray},
    keywordstyle=\color{blue},
    commentstyle=\color{green!50!black},
    stringstyle=\color{red},
    showstringspaces=false,
    escapeinside={(*}{*)}
]
\documentclass{article}
\[\usepackage{xcolor}
\textbf{加粗文本} \\
\textit{斜体文本} \\
]
\end{lstlisting}
\end{description}

% 实例2.9:页眉页脚设置
\item \textbf{页眉页脚设置}
\begin{description}[
leftmargin=7em,
labelwidth=5em,
labelsep=1em,
itemsep=1\parskip,
align=right
]
\item [\textbf{Q}:] 自定义文档的页眉和页脚内容
\item [\textbf{A}:] 
\begin{lstlisting}[
    language={[LaTeX]TeX},
    caption={页眉页脚设置},
    label=code:latex-example,
    basicstyle=\ttfamily\small,
    breaklines=true,
    frame=single,
    numbers=left,
    numberstyle=\tiny\color{gray},
    keywordstyle=\color{blue},
    commentstyle=\color{green!50!black},
    stringstyle=\color{red},
    showstringspaces=false,
    escapeinside={(*}{*)}
]
\documentclass{article}
\[\usepackage{fancyhdr}
\pagestyle{fancy} % 使用fancy页眉页脚样式
\fancyhead[L]{左侧页眉} % 左侧页眉内容    
]
\end{lstlisting}
\end{description}

% 实例2.10:幻灯片制作
\item \textbf{幻灯片制作}
\begin{description}[
leftmargin=7em,
labelwidth=5em,
labelsep=1em,
itemsep=1\parskip,
align=right
]
\item [\textbf{Q}:] 使用Beamer类创建一个简单的幻灯片演示文稿
\item [\textbf{A}:] 
\begin{lstlisting}[
    language={[LaTeX]TeX},
    caption={幻灯片制作},
    label=code:latex-example,
    basicstyle=\ttfamily\small,
    breaklines=true,
    frame=single,
    numbers=left,
    numberstyle=\tiny\color{gray},
    keywordstyle=\color{blue},
    commentstyle=\color{green!50!black},
    stringstyle=\color{red},
    showstringspaces=false,
    escapeinside={(*}{*)}
]
\documentclass{article}
\[\documentclass{beamer}
\usetheme{default} % 使用默认主题

% 演示文稿信息
\title{简单的两页演示文稿}
\author{您的姓名}
\date{\today}

\begin{document}

% 第一页:标题页
\begin{frame}
    \titlepage
\end{frame}

% 第二页:内容页
\begin{frame}{主要内容}
    这是一个简单的两页Beamer演示文稿。
    
    \begin{itemize}
        \item 第一项内容
        \item 第二项内容
        \item 第三项内容
    \end{itemize}
    
    \vspace{0.5cm}
    
\end{frame}

\end{document}]
\end{lstlisting}

\end{description}
\end{enumerate}
\end{document}