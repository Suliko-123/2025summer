\documentclass[a4paper]{article}
\usepackage{ctex}
\ctexset{
    proofname = \heiti{证明}
}
\usepackage{amsmath, amssymb, amsthm}
\usepackage{float} % 提供[H]固定位置选项
\usepackage{moreenum}
\usepackage{mathtools}
\usepackage{url}
\usepackage{bm}
\usepackage{enumitem}
\usepackage{graphicx} % 只保留一次加载
\usepackage{subcaption}
\usepackage{booktabs}
\usepackage{listings}
\usepackage{xcolor}
\usepackage{setspace}
\usepackage{geometry}
\usepackage[mathcal]{eucal}
\usepackage[thehwcnt = 1]{iidef}

% 设置图片文件夹路径(建议使用相对路径,避免绝对路径移植问题)
\graphicspath{{c:/Users/吴虹霖/Desktop/2025小学期/系统开发工具基础/一些图片/}}

% 页面边距配置
\geometry{a4paper, top=2.5cm, bottom=2.5cm, left=3cm, right=2.5cm}
% 全局行间距1.2倍
\linespread{1.2}

\thecourseinstitute{中国海洋大学信息科学与工程学部}
\thecoursename{系统开发工具基础}
\theterm{2025年夏季学期}
\hwname{第一周实验报告}
\slname{\heiti{解}}

\begin{document}
\courseheader
\name{24计算机科学与技术 吴虹霖 \quad 学号: 24020007135}

% ------------------- 第一大板块:Git 练习实例 -------------------
\section{一、Git 练习实例}
\begin{enumerate}[itemsep=2\parskip, label=实例1.\arabic*]

    % 实例1.1:Git命令和数据模型相关内容
\item \textbf {Git 命令和数据模型相关内容}
\begin{description}[
leftmargin=7em,
labelwidth=5em,
labelsep=1em,
itemsep=1\parskip,
align=right
]
\item [\textbf {Q}:] 如果您之前从来没有用过 Git,推荐您阅读 Pro Git 的前几章,或者完成像 Learn Git Branching 这样的教程。重点关注 Git 命令和数据模型相关内容;
\item [\textbf {A}:] 1. 数据模型:Git 通过 “工作区→暂存区→本地版本库” 三层结构管理代码,分支本质是指向特定提交记录的引用,所有历史版本通过提交对象链串联,形成可追溯的变更记录。\\
2. 核心命令:围绕数据流转和分支操作,\texttt {git add} 将工作区文件移至暂存区,\texttt {git commit} 将暂存区内容提交到版本库形成历史记录;\texttt {git branch} 创建分支,\texttt {git checkout} 切换分支,\texttt {git merge} 合并分支,支持多线开发与历史整合。
\end{description}

% 实例1.2:历史版本可视化
\item \textbf {历史版本可视化} % 修正了标题重复问题
\begin{description}[
leftmargin=7em,
labelwidth=5em,
labelsep=1em,
itemsep=1\parskip,
align=right
]
\item [\textbf {Q}:] 克隆本课程网站的仓库,将版本历史可视化并进行探索
\item [\textbf {A}:] 
step1:克隆仓库  
\begin{figure}[H] % 使用[H]固定位置,避免浮动
  \centering
  \includegraphics[width=0.8\textwidth]{克隆仓库.png}
  \caption{克隆仓库截图}
  \label{fig:clone-repo} % 唯一标签
\end{figure}  

step2:可视化版本历史
\begin{figure}[H]
  \centering
  \includegraphics[width=0.8\textwidth]{查看历史版本.png}
  \caption{查看历史版本截图}
  \label{fig:view-history} % 唯一标签
\end{figure} 
\end{description}

    % 实例1.3:查看修改
\item \textbf {查看修改}
\begin{description}[
leftmargin=7em,
labelwidth=5em,
labelsep=1em,
itemsep=1\parskip,
align=right
]
\item [\textbf {Q}:] 是谁最后修改了 README.md 文件?(提示:使用 git log 命令并添加合适的参数)
\item [\textbf {A}:] 如图所示。
\begin{figure}[H]
  \centering
  \includegraphics[width=0.8\textwidth]{查看修改人.png}
  \caption{查看修改人截图}
  \label{fig:view-modifier} % 唯一标签
\end{figure}
\end{description}

    % 实例1.4:查看文件修改与详细信息
\item \textbf {查看文件修改与详细信息}
\begin{description}[
leftmargin=7em,
labelwidth=5em,
labelsep=1em,
itemsep=1\parskip,
align=right
]
\item [\textbf {Q}:] 最后一次修改\_config.yml 文件中 collections: 行时的提交信息是什么?(提示:使用 git blame 和 git show)
\item [\textbf {A}:] 如图所示。
\begin{figure}[H] % 修正为figure环境(原itemize错误)
  \centering
  \includegraphics[width=0.8\textwidth]{查看最后一次修改记录.png}
  \caption{查看最后一次修改记录截图}
  \label{fig:last-modify-record} % 唯一标签
\end{figure}

\begin{figure}[H] % 修正为figure环境(原itemize错误)
  \centering
  \includegraphics[width=0.8\textwidth]{查看提交详细信息.png}
  \caption{查看提交详细信息截图}
  \label{fig:commit-details} % 唯一标签
\end{figure}
\end{description}

    % 实例1.5:提交添加信息并删除
\item \textbf {提交添加信息并删除}
\begin{description}[
leftmargin=7em,
labelwidth=5em,
labelsep=1em,
itemsep=1\parskip,
align=right
]
\item [\textbf {Q}:] 使用 Git 时的一个常见错误是提交本不应该由 Git 管理的大文件,或是将含有敏感信息的文件提交给 Git 。尝试向仓库中添加一个文件并添加提交信息,然后将其从历史中删除
\item [\textbf {A}:] 如图所示。
\begin{figure}[H]
  \centering
  \includegraphics[width=0.8\textwidth]{添加敏感信息.png}
  \caption{添加敏感信息}
  \label{fig:view-modifier} % 唯一标签
\end{figure}

\begin{figure}[H]
  \centering
  \includegraphics[width=0.8\textwidth]{清理提交信息.png}
  \caption{清理提交信息.png}
  \label{fig:view-modifier} % 唯一标签
\end{figure}
\end{description}

    % 实例1.6:git stash
\item \textbf {git stash}
\begin{description}[
leftmargin=7em,
labelwidth=5em,
labelsep=1em,
itemsep=1\parskip,
align=right
]
\item [\textbf {Q}:] 从 GitHub 上克隆某个仓库,修改一些文件。当您使用 git stash 会发生什么?
\item [\textbf {A}:] 执行 git stash 后,添加到暂存区的内容不会再提示需要提交(Changes to be committed)。且HEAD引用不会变动。
\end{description}

% 实例1.7:别名
\item \textbf {别名}
\begin{description}[
leftmargin=7em,
labelwidth=5em,
labelsep=1em,
itemsep=1\parskip,
align=right
]
\item [\textbf {Q}:] 请在 ~/.gitconfig 中创建一个别名,使您在运行 git graph 时,您可以得到 git log --all --graph --decorate --oneline 的输出结果
\item [\textbf {A}:] 
\end{description}











\end{enumerate}
\end{document}

