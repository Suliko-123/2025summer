\documentclass[a4paper]{article}
\usepackage{ctex}
\ctexset{
    proofname = \heiti{证明}
}
\usepackage{amsmath, amssymb, amsthm}
\usepackage{float}
\usepackage{moreenum}
\usepackage{mathtools}
\usepackage{url}
\usepackage{bm}
\usepackage{enumitem}
\usepackage{graphicx}
\usepackage{subcaption}
\usepackage{booktabs}
\usepackage{listings}  % 代码块高亮需要
\usepackage{xcolor}    % 代码块颜色需要
\usepackage{setspace}  % 行间距需要
\usepackage{geometry}  % 页面边距需要
\usepackage[mathcal]{eucal}
\usepackage[thehwcnt = 1]{iidef}
% 导言区(已加载 graphicx 包的前提下)
\usepackage{graphicx}
% 设置图片文件夹路径(用 // 或 / 分隔)
\graphicspath{{c:/Users/吴虹霖/Desktop/2025小学期/系统开发工具基础/一些图片/}}

% 页面边距配置(符合学术规范)
\geometry{a4paper, top=2.5cm, bottom=2.5cm, left=3cm, right=2.5cm}
% 全局行间距1.2倍
\linespread{1.2}

\thecourseinstitute{中国海洋大学信息科学与工程学部}
\thecoursename{系统开发工具基础}
\theterm{2025年夏季学期}
\hwname{第一周实验报告}
\slname{\heiti{解}}

\begin{document}
\courseheader
\name{24计算机科学与技术 吴虹霖 \quad 学号: 24020007135}

% ------------------- 第一大板块:Git 练习实例 -------------------
\section{一、Git 练习实例}
\begin{enumerate}[itemsep=2\parskip, label=实例1.\arabic*]

    % 实例1.1:Git命令和数据模型相关内容
\item \textbf {Git 命令和数据模型相关内容}
\begin{description}[
leftmargin=7em, % 整体左缩进,为标签预留空间
labelwidth=5em, % 标签固定宽度(Q 和 A 对齐的关键)
labelsep=1em, % 标签与内容的间距
itemsep=1\parskip, % Q 和 A 之间的垂直间距
align=right % 标签右对齐,增强视觉对齐感
]
\item [\textbf {Q}:] 如果您之前从来没有用过 Git,推荐您阅读 Pro Git 的前几章,或者完成像 Learn Git Branching 这样的教程。\
重点关注 Git 命令和数据模型相关内容;
\item [\textbf {A}:] 1. 数据模型:Git 通过 “工作区→暂存区→本地版本库” 三层结构管理代码,分支本质是指向特定提交记录的引用,所有历史版本通过提交对象链串联,形成可追溯的变更记录。\
\\2. 核心命令:围绕数据流转和分支操作,\texttt {git add} 将工作区文件移至暂存区,\texttt {git commit} 将暂存区内容提交到版本库形成历史记录;\texttt {git branch} 创建分支,\texttt {git checkout} 切换分支,\texttt {git merge} 合并分支,支持多线开发与历史整合。
\end{description}

% 实例1.2:历史版本可视化
\item \textbf {Git 命令和数据模型相关内容}
\begin {description}[
leftmargin=7em, % 整体左缩进,为标签预留空间
labelwidth=5em, % 标签固定宽度(Q 和 A 对齐的关键)
labelsep=1em, % 标签与内容的间距
itemsep=1\parskip, % Q 和 A 之间的垂直间距
align=right % 标签右对齐,增强视觉对齐感
]
\item [\textbf {Q}:] 克隆本课程网站的仓库,将版本历史可视化并进行探索
\item [\textbf {A}:] 
step1:克隆仓库  
\begin{figure}[hbtp]  % h=当前位置, b=页底, t=页顶, p=单独页
  \centering  % 图片居中
  \includegraphics[width=0.8\textwidth]{克隆仓库.png}  % 图片宽度设为页面的80%
  \caption{克隆仓库截图}  % 图片标题
  \label{fig:git-log}  % 图片标签(用于文中引用,如“图\ref{fig:git-log}”)
\end{figure}  

step2:可视化版本历史

\begin{figure}[hbtp]  % h=当前位置, b=页底, t=页顶, p=单独页
  \centering  % 图片居中
  \includegraphics[width=0.8\textwidth]{查看历史版本.png}  % 图片宽度设为页面的80%
  \caption{查看历史版本截图}  % 图片标题
  \label{fig:git-log}  % 图片标签(用于文中引用,如“图\ref{fig:git-log}”)
\end{figure} 
\end{description}

    % 实例1.3:查看修改
\item \textbf {查看修改}
\begin {description}[
leftmargin=7em, % 整体左缩进,为标签预留空间
labelwidth=5em, % 标签固定宽度(Q 和 A 对齐的关键)
labelsep=1em, % 标签与内容的间距
itemsep=1\parskip, % Q 和 A 之间的垂直间距
align=right % 标签右对齐,增强视觉对齐感
]
\item [\textbf {Q}:] 是谁最后修改了 README.md 文件?(提示:使用 git log 命令并添加合适的参数)
\item [\textbf {A}:] 如图所示。\\
\begin{figure}[H]
  \centering
  \includegraphics[width=0.8\textwidth]{查看修改人.png}
  \caption{查看修改人}
  \label{fig:modify-person}
\end{figure}
\end{description}

% 实例1.4:Git远程仓库关联(GitHub/Gitee)
\item \textbf{Git远程仓库关联(GitHub/Gitee)}
  \begin{itemize}[leftmargin=2em, itemsep=0.5\parskip]
    \item \textbf{操作目的}:实现本地仓库与远程仓库(如GitHub)的连接,推送本地代码到远程。
    \item \textbf{操作步骤}:
      1. 在GitHub创建新仓库(如 \texttt{git-practice-2025},不勾选“Initialize this repository with a README”);
      2. 本地终端执行 \texttt{git remote add origin https://github.com/你的用户名/git-practice-2025.git}(关联远程仓库);
      3. 执行 \texttt{git push -u origin main}(推送主分支到远程,首次推送需输入GitHub账号密码/Token);
      4. 浏览器打开GitHub仓库页面,验证是否看到本地提交的 \texttt{readme.md}。
    \item \textbf{结果说明}:终端显示“Writing objects: 100% ...”(推送成功),GitHub仓库页面能看到本地代码。
  \end{itemize}

% 实例1.5:Git拉取远程仓库更新
\item \textbf{Git拉取远程仓库更新}
  \begin{itemize}[leftmargin=2em, itemsep=0.5\parskip]
    \item \textbf{操作目的}:学会从远程仓库拉取他人/自己在其他设备的更新,保持本地与远程同步。
    \item \textbf{操作步骤}:
      1. 在GitHub网页端直接修改 \texttt{readme.md}(如添加“remote update: 2025-08-31”),并提交;
      2. 本地终端执行 \texttt{git pull origin main}(拉取远程主分支的更新);
      3. 打开本地 \texttt{readme.md},查看是否包含网页端添加的内容。
    \item \textbf{结果说明}:终端显示“Updating a1b2c3d..e4f5g6h”(拉取成功),本地文件与远程完全同步。
  \end{itemize}

% 实例1.6:Git撤销暂存区文件
\item \textbf{Git撤销暂存区文件}
  \begin{itemize}[leftmargin=2em, itemsep=0.5\parskip]
    \item \textbf{操作目的}:掌握“将文件从暂存区撤回工作区”的操作(误add后补救)。
    \item \textbf{操作步骤}:
      1. 本地创建新文件 \texttt{test.txt},写入“test file”;
      2. 执行 \texttt{git add test.txt}(将文件加入暂存区);
      3. 执行 \texttt{git status}(查看状态,显示“Changes to be committed: test.txt”);
      4. 执行 \texttt{git rm --cached test.txt}(撤销暂存区的 \texttt{test.txt});
      5. 再次执行 \texttt{git status},验证是否显示“Untracked files: test.txt”(回到未暂存状态)。
    \item \textbf{结果说明}:\texttt{test.txt} 从“待提交”状态变回“未追踪”状态,工作区文件未被删除。
  \end{itemize}

% 实例1.7:Git查看历史提交记录(筛选与格式化)
\item \textbf{Git查看历史提交记录(筛选与格式化)}
  \begin{itemize}[leftmargin=2em, itemsep=0.5\parskip]
    \item \textbf{操作目的}:高效筛选历史记录(如指定作者、时间),简化输出格式。
    \item \textbf{操作步骤}:
      1. 执行 \texttt{git log --oneline}(简化显示:短哈希值+提交信息,一行一条);
      2. 执行 \texttt{git log --author="你的用户名"}(只显示你提交的记录);
      3. 执行 \texttt{git log --since="2025-08-30" --until="2025-08-31"}(显示指定日期范围内的记录);
      4. 执行 \texttt{git log -p readme.md}(查看 \texttt{readme.md} 文件的所有修改历史,含具体内容差异)。
    \item \textbf{结果说明}:不同命令对应不同筛选结果,能快速定位目标提交记录。
  \end{itemize}

% 实例1.8:Git解决简单合并冲突
\item \textbf{Git解决简单合并冲突}
  \begin{itemize}[leftmargin=2em, itemsep=0.5\parskip]
    \item \textbf{操作目的}:处理“同一文件同一行被不同分支修改”导致的合并冲突。
    \item \textbf{操作步骤}:
      1. 切到 \texttt{dev} 分支:\texttt{git checkout dev},修改 \texttt{readme.md} 第1行为“dev: conflict test”,提交;
      2. 切回 \texttt{main} 分支:\texttt{git checkout main},修改 \texttt{readme.md} 第1行为“main: conflict test”,提交;
      3. 执行 \texttt{git merge dev},终端显示“Automatic merge failed; fix conflicts and then commit the result.”(冲突提示);
      4. 打开 \texttt{readme.md},看到冲突标记:
         \begin{verbatim}
         <<<<<<< HEAD (当前分支:main)
         main: conflict test
         =======
         dev: conflict test
         >>>>>>> dev (待合并分支:dev)
         \end{verbatim}
      5. 删除冲突标记,修改内容为“fixed conflict: main + dev”,执行 \texttt{git add readme.md \&\& git commit -m "fix merge conflict"}(提交解决冲突);  % 修复:&&→\&\&
    \item \textbf{结果说明}:合并成功,\texttt{readme.md} 保留修改后的内容,\texttt{git log} 能看到“fix merge conflict”的提交记录。
  \end{itemize}

% 实例1.9:Git删除本地与远程分支
\item \textbf{Git删除本地与远程分支}
  \begin{itemize}[leftmargin=2em, itemsep=0.5\parskip]
    \item \textbf{操作目的}:清理无用分支(本地+远程),保持仓库整洁。
    \item \textbf{操作步骤}:
      1. 切回 \texttt{main} 分支:\texttt{git checkout main}(删除分支前需退出该分支);
      2. 执行 \texttt{git branch -d dev}(删除本地 \texttt{dev} 分支,若分支有未合并修改,用 \texttt{-D} 强制删除);
      3. 执行 \texttt{git push origin --delete dev}(删除远程 \texttt{dev} 分支);
      4. 执行 \texttt{git branch}(验证本地 \texttt{dev} 已删除),访问GitHub仓库查看“Branches”,验证远程 \texttt{dev} 已删除。
    \item \textbf{结果说明}:本地和远程的 \texttt{dev} 分支均被删除,仅保留 \texttt{main} 分支。
  \end{itemize}

% 实例1.10:Git忽略文件(.gitignore配置)
\item \textbf{Git忽略文件(.gitignore配置)}
  \begin{itemize}[leftmargin=2em, itemsep=0.5\parskip]
    \item \textbf{操作目的}:让Git自动忽略不需要追踪的文件(如日志、缓存、IDE配置文件)。
    \item \textbf{操作步骤}:
      1. 本地创建 \texttt{log.txt}(日志文件)和 \texttt{vscode/} 文件夹(IDE配置文件夹);
      2. 在仓库根目录创建 \texttt{.gitignore} 文件,写入以下内容:
         \begin{verbatim}
         # 忽略日志文件
         log.txt
         # 忽略VSCode配置文件夹
         vscode/
         # 忽略所有.csv文件
         *.csv
         \end{verbatim}
      3. 执行 \texttt{git add .gitignore \&\& git commit -m "add .gitignore: ignore log/vscode/csv"}(提交配置);  % 修复:&&→\&\&
      4. 执行 \texttt{git status},验证 \texttt{log.txt} 和 \texttt{vscode/} 是否显示“Untracked files”(Git已忽略它们,不提示追踪)。
    \item \textbf{结果说明}:\texttt{log.txt}、\texttt{vscode/} 等文件不会被Git追踪,避免提交无用文件到仓库。
  \end{itemize}

\end{enumerate}

\end{document}


%%% Local Variables:
%%% mode: latex
%%% TeX-master: t
%%% End: